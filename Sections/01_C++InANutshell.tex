\section{C++ in a nutshell}
This is just a collection of sparse hints and pointers to important topics. For a complete reference and introduction to C++ and its features, the lecturer suggests referring to the book \emph{Absolute C++}, from Walter Savitch.

\subsection{Data types}
C++ data types are \emph{bool}, \emph{int}, \emph{char}, \emph{float}, \emph{double} and their unsigned variants, plus \emph{string}.

Explicit type casts can be performed either in \emph{C style}, e.g. \code{int a = (int)f + b;}, or better in \emph{C++ style}, e.g. \code{int a = static_cast<int>(f) + b;}.
However special attention must be paid to \emph{implicit casting}, as e.g. in \code{int a = f + b;} which is equivalent to \code{int a = static_cast<int>(f + b);}.

It goes without saying that, in properly designed and written code, casts should be avoided wherever possible.

\subsection{Functions}
Functions in C++ have:
\begin{description}
	\item[Declaration] Which define name, parameters and return type and must go before any usage.
	\item[Definition] The actual function code: it may go after usages.
	\item[Signature] The number and types of parameters and their qualifiers (as e.g. \code{const}) is called the \emph{signature} of a function\footnote{The return type and the name are not part of the signature.}.
\end{description}
\begin{lstlisting}[float,frame=tb,caption={Example of declaration, usage and definiton of function.}]
	#include<iostream>
	double sum(double a, double b); // declaration
	int main()
	{
		double a = 1, b = 2, c;
		c = sum(a,b); // usage
		std::cout << c << std::endl;
	}
	double sum(double a, double b) // definition
	{
		return a + b;
	}
\end{lstlisting}
In C++ it is possible to perform \emph{overloading}, i.e. to declare and define multiple functions with the same name but different signatures and return types. Please note that the differentiation is due to having a different signature, not a different return type: this means that obviously we can't overload a function just by changing its return type but leaving the same signature, as the compiler would not be able to know which one to use.

\subsection{Pre-compiler directives}
The purpose of pre-compiler directives is to define special behaviors or variables at compile time.
It is used mainly for template metaprogramming, to avoid inclusion loops with header files and specific directives such as macros.
An example of pre-compiler directive is \code{#define D 3}, which defines a substitution of each single uppercase \code{D} to be substituted by the const int \code{3} at compile time. This could be useful for defining parametric array lenghts which may be tuned at compile time.

\subsection{Pointers and references}
Basically a pointer contains the address of a memory block, so pointers can be used to store addresses of variables. E.g.:
\begin{lstlisting}
	int a = 6;
	int* b = &a;
	std::cout << b << std::endl; // print the address of a
	std::cout << *b << std::endl; // print "6"
\end{lstlisting}
A pointer is nothing more than a long integer representing the memory address: this means that arithmetic operations can be performed on pointers in order to access different memory blocks according to specific patterns.

C++ also has the concept of \emph{reference}. A reference to some variable is like a variable itself (access-wise), however if passed as argument to a function, it produces a pass-by-reference behaviour. E.g.:
\begin{lstlisting}
	double d = 3.14;
	double& rd = d;
	std::cout << rd << std::endl; // print "3.14"
\end{lstlisting}
References can be effectively used to achieve a pass-by-reference behaviour, without having the possible downsides of using pointers.

\subsection{Memory allocation}
In C++ the standard way to allocate and free memory at runtime is by using the \code{new} and \code{delete} keywords:
\begin{lstlisting}
	int* i = new int(5); // int with value = 5
	int* a = new int[5]; // array of 5 uninitialized ints
	delete i; // free mem of integer
	delete [] a; // free mem of array
\end{lstlisting}
It is also possible to use the old \code{malloc}, \code{calloc}, \code{realloc} and \code{free} C keywords for backward compatibility, however it must be kept in mind that these will not trigger the call of constructors and destructors on objects, as instead \code{new} and \code{delete} do.

A \code{new} must always match a corresponding \code{delete}, otherwise we end up allocating more memory than we will eventually free: this situation is called \emph{memory leaking}.

\paragraph{Stack vs. Heap}
It is worth remembering that variables/objects allocated in stack are temporary and local in scope, while those allocated in heap, by using \code{new} can be accessed globally (as long as their pointers are passed around) and must be explicitly deleted by the developer.
%EOF

